\documentclass[sigconf, authorversion]{acmartreview}

\usepackage{booktabs} % For formal tables
\usepackage[acronyms, nogroupskip, nonumberlist, nopostdot]{glossaries}
\makeglossaries
\usepackage{IEEEtrantools}
\usepackage{xargs}
\usepackage{algorithm}
\usepackage{algorithmicx}
\usepackage[noend]{algpseudocode}
\usepackage[inline,shortlabels]{enumitem}
\usepackage[binary-units=true]{siunitx}
\usepackage{subcaption}
\usepackage{url}
\usepackage{xcolor}

% TODO notes
\usepackage{todonotes}
 \presetkeys{todonotes}{inline, backgroundcolor=green, linecolor=black}{}
\newcommand{\michael}[1]{\todo[backgroundcolor=green]{Michael: #1}}
\newcommand{\paul}[1]{\todo[backgroundcolor=yellow]{Paul: #1}}
\newenvironment*{modifiedparagraph}{}{}
\newcommand{\modifiedtext}[1]{#1}

% Copyright
\setcopyright{none}
%\setcopyright{acmcopyright}
%\setcopyright{acmlicensed}
%\setcopyright{rightsretained}
%\setcopyright{usgov}
%\setcopyright{usgovmixed}
%\setcopyright{cagov}
%\setcopyright{cagovmixed}

\settopmatter{printacmref=false, printccs=false, printfolios=true}

\hyphenation{post-proces-sing}

% DOI
%\acmDOI{10.475/123_4}

% ISBN
%\acmISBN{123-4567-24-567/08/06}

%Conference
%\acmConference[WOODSTOCK'97]{ACM Woodstock conference}{July 1997}{El
%  Paso, Texas USA} 
%\acmYear{1997}
%\copyrightyear{2016}
%
%\acmPrice{15.00}


\begin{document}
\title{GRASP-based Gate-Control List Synthesis for IEEE~Time-Sensitive Networks (TSN)}

\author{Michael L. Raagaard and Paul Pop}
\affiliation{%
  \institution{Technical University of Denmark}
  \city{Kongens Lyngby}
  \country{Denmark}
}
\email{michael@raagaard.dk,\quad paupo@dtu.dk}

\author{Silviu S. Craciunas}
\affiliation{%
  \institution{TTTech Computertechnik AG}
  \city{Vienna} 
  \country{Austria} 
}
\email{scr@tttech.com}

\begin{abstract}
\gls{tsn} is an IEEE effort to bring deterministic real-time capabilities to \IEEE{802.3 Ethernet}. \gls{tsn} meets the bandwidth and dependability requirements of emerging mixed-criticality applications, while ensuring timeliness of time-critical communication. The \IEEE{802.1Qbv} sub-standard introduces time-aware gates within network devices enabling fully deterministic temporal behavior of real-time communication. For each egress port, a \gls{gcl} specifies which queue (traffic class) may transmit at which points in time. Using this functionality enables frames to be forwarded in the network in a time-triggered manner.
In this paper, we are interested in synthesizing schedule tables for periodic time-critical flows on \gls{tsn} architectures such that timeliness is guaranteed, i.e., an assignment of critical Ethernet frames to egress port queues, and a construction of \glsplural{gcl} ensuring that all frames are schedulable.
The problem is formulated as a multi-objective combinatorial optimization problem, which minimizes the queue utilization as well as end-to-end latency.
We propose a scalable \gls{grasp}-based metaheuristic to solve this problem. The proposed approach is evaluated on synthetic and realistic test cases, and compared to results from related work.
\end{abstract}

%% MACROS 
\newcommand{\IEEE}[1]{\textit{IEEE #1}}
\newcommand{\mus}[1]{\SI{#1}{\micro\second}}
\renewcommand{\ms}[1]{\SI{#1}{\milli\second}}
\renewcommandx{\l}[1][1={v_a,v_b}]{\ensuremath{\left[#1 \right]}}
\newcommandx{\f}[3][1={i,m}, 2={\l}, 3=f]{\ensuremath{#3_{#1}^{#2}}}
\renewcommandx{\s}[3][1={i}, 2={}, 3=s]{\ensuremath{#3_{#1}^{#2}}}
\newcommand{\Kappa}{\mathrm{K}}
\newcommand{\lbound}[1]{\ensuremath{\underline{#1}}}
\newcommand{\ubound}[1]{\ensuremath{\overline{#1}}}

%autoref
\renewcommand*{\figureautorefname}{Fig.}
\renewcommand*{\sectionautorefname}{Sect.}
\renewcommand*{\subsectionautorefname}{Sect.}
\renewcommand*{\subsubsectionautorefname}{Sect.}
\renewcommand*{\equationautorefname}{Eq.}
\renewcommand*{\tableautorefname}{Table}
\newcommand{\algorithmautorefname}{Alg.}

\setlength{\abovecaptionskip}{0.5ex}
\setlength{\belowcaptionskip}{0.3ex}
%\setlength{\floatsep}{10ex}
\setlength{\textfloatsep}{2ex}
%\setlength{\bibsep}{0ex}
%\setlength{\partopsep}{0.3ex}

% glossary
% Abbreviations:
\newacronym{tsn}{TSN}{Time-Sensitive Networking}
\newacronym{tas}{TAS}{Time Aware Shaper}
\newacronym{cbs}{CBS}{Credit Based Shaper}
\newacronym{mtu}{MTU}{Maximum Transmission Unit}
\newacronym{tt}{TT}{Time-Triggered}
\newacronym{avb}{AVB}{Audio-Video Bridging}
\newacronym{be}{BE}{Best-Effort}
\newacronym{gcl}{GCL}{Gate-Control List}
\newacronym{qos}{QoS}{Quality of Service}
\newacronym{lcm}{LCM}{Least Common Multiple}
\newacronym{pcp}{PCP}{Priority Code Point}
\newacronym{vid}{VID}{VLAN Identifier}
\newacronym{fifo}{FIFO}{first in, first out}
\newacronym{ilp}{ILP}{Integer Linear Programming}
\newacronym{asap}{ASAP}{as-soon-as-possible}
\newacronym{alap}{ALAP}{as-late-as-possible}
\newacronym{grasp}{GRASP}{Greedy Randomized Adaptive Search Procedure}
\newacronym{smt}{SMT}{Satisfiability Modulo Theories}
\newacronym{omt}{OMT}{Optimization Modulo Theories}
\newacronym{rcl}{RCL}{Restricted Candidate List}
\newacronym{mip}{MIP}{Mixed Integer Programming}
\newacronym{wcd}{WCD}{Worst-Case Delay}
\newacronym{iqr}{IQR}{Interquartile Range}
\newacronym{edf}{EDF}{Earliest Deadline First}

% Defining non-standard plural:
% \newglossaryentry{LED}
% {
%   name={LED},
%   description={light emitting diode},
%   first={\glsentrydesc{LED} (\glsentrytext{LED})},
%   plural={LEDs},
%   descriptionplural={light emitting diodes},
%   firstplural={\glsentrydescplural{LED} (\glsentryplural{LED})}
% } 

%\setglossarystyle{myabbrev}
%\printglossary[type=\acronymtype, title=Abbreviations]                                   
% Architecture Model:
\newglossaryentry{network}{
    name = {\ensuremath{\mathcal{G}(\mathbf{V},\mathbf{E})}},
    description = {Directed graph of network topology},
    sort = {G_E_V},
}

\newglossaryentry{network_short}{
    name = {\ensuremath{\mathcal{G}}},
    description = {Directed graph of network topology},
    sort = {G},
}

\newglossaryentry{network_v}{
    name = {\ensuremath{\mathbf{V}}},
    description = {Devices in network},
    sort = {V},
}
\newglossaryentry{network_e}{
    name = {\ensuremath{\mathbf{E}}},
    description = {Data links in network},
    sort = {E},
}

\newglossaryentry{es_set}{
    name = {\ensuremath{\mathbf{ES}}},
    description = {End systems in network},
    sort = {ES},
}

\newglossaryentry{sw_set}{
    name = {\ensuremath{\mathbf{SW}}},
    description = {Switches in network},
    sort = {SW},
}

\newglossaryentry{device}{
    name = {\ensuremath{v_a}},
    description = {Network device, either switch or end system},
    sort = {v_a},
}

\newglossaryentry{link}{
    name = {\l},
    description = {Data link from $v_a$ to $v_b$},
    sort = {v_a_v_b},
}

\newglossaryentry{link_transmission_rate}{
    name = {\ensuremath{\l.s}},
    description = {Transmission rate on link \l},
    sort = {v_a_v_b_s},
}

\newglossaryentry{link_propagation_delay}{
    name = {\ensuremath{\l.d}},
    description = {Propagation delay on link \l},
    sort = {v_a_v_b_d},
}

\newglossaryentry{link_queues}{
    name = {\ensuremath{\l.c}},
    description = {Number of queues associated with egress port \l},
    sort = {v_a_v_b_c},
}

\newglossaryentry{route_set}{
    name = {\ensuremath{R_{a,b}}},
    description = {Set of all routes from $v_a$ to $v_b$},
    sort = {R_ab},
}

\newglossaryentry{route}{
    name = {\ensuremath{r_k}},
    description = {Route in network},
    sort = {r_k},
}

\newglossaryentry{synch_error}{
    name = {$\delta$},
    description = {Maximum synchronization error},
    sort = {d},
}

% Application Model:
\newglossaryentry{flows_set}{
    name = {\ensuremath{\mathcal{S}}},
    description = {Set of all flows},
    sort = {S},
}

\newglossaryentry{heuristics_set}{
    name = {\ensuremath{\mathcal{H}}},
    description = {Set of all flows},
    sort = {S},
}

\newglossaryentry{frames_set}{
    name = {\ensuremath{\mathcal{F}}},
    description = {Set of all frames},
    sort = {F},
}

\newglossaryentry{flow_instances_set}{
    name = {\ensuremath{\mathcal{I}}},
    description = {Set of all flow instances},
    sort = {I},
}

\newglossaryentry{flow}{
    name = {\s},
    description = {A single flow},
    sort = {s_i},
}

\newglossaryentry{flow_T}{
    name = {\ensuremath{\s.T}},
    description = {Period of \s},
    sort = {s_i_T},
}

\newglossaryentry{flow_D}{
    name = {\ensuremath{\s.D}},
    description = {Deadline of \s},
    sort = {s_i_D},
}

\newglossaryentry{flow_va}{
    name = {\ensuremath{\s.v_a}},
    description = {Sending end system for \s},
    sort = {s_i_v_a},
}

\newglossaryentry{flow_vb}{
    name = {\ensuremath{\s.v_b}},
    description = {Receiving end system for \s},
    sort = {s_i_v_b},
}

\newglossaryentry{flow_r}{
    name = {\ensuremath{\s.r}},
    description = {Route of \s},
    sort = {s_i_r},
}

\newglossaryentry{flow_s}{
    name = {\ensuremath{\s.s}},
    description = {Sending link of \s},
    sort = {s_i_s},
}

\newglossaryentry{flow_t}{
    name = {\ensuremath{\s.t}},
    description = {Receiving link of \s},
    sort = {s_i_t},
}

\newglossaryentry{flow_k}{
    name = {\ensuremath{\s.k}},
    description = {Number of frames on each link of \s},
    sort = {s_i_k},
}

\newglossaryentry{flow_I}{
    name = {\ensuremath{\s.I}},
    description = {Set of flow instances associated with \s},
    sort = {s_i_I},
}

\newglossaryentry{flow_instance}{
    name = {\s[i][\l]},
    description = {Instance of flow \s{} on link \l},
    sort = {s_i_v_a_v_b},
}

\newglossaryentry{flow_instance_rho}{
    name = {\ensuremath{\s[i][\l].\rho}},
    description = {Queue assignment for \s[i][\l]},
    sort = {s_i_v_a_v_b_r},
}

\newglossaryentry{flow_instance_F}{
    name = {\ensuremath{\s[i][\l].F}},
    description = {Frames associated with \s[i][\l]},
    sort = {s_i_v_a_v_b_F},
}

\newglossaryentry{frame}{
    name = {\f},
    description = {A single frame},
    sort = {f_im_v_a_v_b},
}

\newglossaryentry{frame_L}{
    name = {\ensuremath{\f.L}},
    description = {Transmission duration of \f},
    sort = {f_im_v_a_v_b.L},
}

\newglossaryentry{frame_phi}{
    name = {\ensuremath{\f.\phi}},
    description = {Periodic offset of \f},
    sort = {f_im_v_a_v_b_ph},
}

% ILP
\newglossaryentry{X}{
    name = {\ensuremath{X}},
    description = {Solution space},
    sort = {X},
}

\newglossaryentry{x}{
    name = {\ensuremath{x}},
    description = {Feasible solution},
    sort = {x},
}

\newglossaryentry{x_opt}{
    name = {\ensuremath{x^*}},
    description = {Optimal solution},
    sort = {x*},
}

\newglossaryentry{z}{
    name = {\ensuremath{z}},
    description = {Objective value},
    sort = {z},
}

\newglossaryentry{z_opt}{
    name = {\ensuremath{z^*}},
    description = {Optimal objective value},
    sort = {z*},
}

\newglossaryentry{Kappa}{
    name = {$\Kappa$},
    description = {Excess queue usage},
    sort = {K},
}

\newglossaryentry{kappa}{
    name = {$\kappa_{a,b}$},
    description = {Number of queues used by TT flows in \l},
    sort = {k},
}

\newglossaryentry{Lambda}{
    name = {$\Lambda$},
    description = {Additional end-to-end latency},
    sort = {L},
}

\newglossaryentry{lambda}{
    name = {$\lambda_i$},
    description = {End-to-end latency for \s},
    sort = {l_i},
}

\newglossaryentry{lambda_lbound}{
    name = {$\lbound{\lambda}_i$},
    description = {Lower bound on end-to-end latency},
    sort = {l_i_lbound},
}

\newglossaryentry{frame_ph_lbound}{
    name = {\ensuremath{\f.\lbound{\phi}}},
    description = {Lower bound on periodic offset},
    sort = {f_im_v_a_v_b_ph_lbound},
}

\newglossaryentry{frame_ph_ubound}{
    name = {\ensuremath{\f.\ubound{\phi}}},
    description = {Upper bound on periodic offset},
    sort = {f_im_v_a_v_b_ph_ubound}
}

\newglossaryentry{A}{
    name = {$A$},
    description = {Set of repetitions of \s[i] within hyperperiod},
    sort = {A},
}

\newglossaryentry{B}{
    name = {$B$},
    description = {Set of repetitions of \s[j] within hyperperiod},
    sort = {B},
}

\newglossaryentry{alpha}{
    name = {$\alpha$},
    description = {Individual repetition of \s[i] within hyperperiod},
    sort = {a},
}

\newglossaryentry{beta}{
    name = {$\beta$},
    description = {Individual repetition of \s[j] within hyperperiod},
    sort = {b},
}

\newglossaryentry{M}{
    name = {$M$},
    description = {Theoretically infinitely large constant},
    sort = {M},
}

\newglossaryentry{sigma}{
    name = {$\sigma$},
    description = {Variable for modeling disjunction},
    sort = {s},
}

\newglossaryentry{epsilon}{
    name = {$\epsilon$},
    description = {Variable for modeling identical queue assignments},
    sort = {e},
}

\newglossaryentry{omega}{
    name = {$\omega$},
    description = {Variable for modeling disjunction},
    sort = {o},
}

% Heuristic
\newglossaryentry{Phi_alpha}{
    name = {$\Phi_{\alpha}$},
    description = {Feasible region in repetition $\alpha$},
    sort = {Ph_alpha},
}

\newglossaryentry{Phi_cap}{
    name = {$\Phi_{\cap}$},
    description = {Feasible regions intersection},
    sort = {Ph_intersection},
}

\newglossaryentry{phi_lbound}{
    name = {$\lbound{\phi}$},
    description = {Function for lower bound on period offsets},
    sort = {ph_lbound},
}

\newglossaryentry{frame_repetitions_set}{
    name = {\ensuremath{\mathcal{R}}},
    description = {Set of all frame repetitions},
    sort = {R},
}

% Metaheuristic
\newglossaryentry{gamma}{
    name = {$\gamma$},
    description = {Length of \gls{rcl}},
    sort = {g},
}

\newglossaryentry{pi}{
    name = {$\pi$},
    description = {Maximum number of flows to destroy},
    sort = {p},
}

\newglossaryentry{l}{
    name = {$l$},
    description = {Ratio of time spent in local search phase},
    sort = {l},
}

\newglossaryentry{a}{
    name = {$a$},
    description = {Local search strategy},
    sort = {a},
}

\newglossaryentry{f}{
    name = {\ensuremath{f(x)}},
    description = {Metaheuristic objective function},
    sort = {f},
}

\newglossaryentry{D_pi}{
    name = {\ensuremath{D_{\pi}(x)}},
    description = {Combinations of choosing $\pi$ link-sharing flows from $x$},
    sort = {D_p},
}

%\setglossarystyle{mynotation}
%\printglossary[title=Notations]

%\vspace{-1cm}

%\glsaddall

\maketitle
\vspace{-0.1cm}
\section{Introduction}
In this paper, we are interested in safety-critical real-time applications implemented using distributed cyber-physical systems. There are several communication protocols on the market, depending on the application area, e.g., FlexRay for automotive, AFDX for avionics, and EtherCAT for industrial automation.
However, emerging applications, e.g., Advanced Driver Assistance Systems (ADAS), autonomous driving, or Industry 4.0, have increasing bandwidth demands.
For instance, autonomous driving requires data rates of at least 100 Mbps for graphical computing based on camera, radar, and LiDAR data, whereas CAN and FlexRay only provide data rates of up to 1 Mbps and 10 Mbps, respectively.

The well-known networking standard \IEEE{802.3 Ethernet}~\cite{8023} meets the emerging bandwidth requirements for safety-critical networks, while remaining scalable and cost-effective.
It does, however, lack real-time and dependability capabilities~\cite{decotignie5}.
Many extensions, such as EtherCAT, PROFINET, ARINC 664p7~\cite{ARINC2009}, and TTEthernet~\cite{ttethernet11}, have been suggested and are used in the industry.
Although they satisfy the timing requirements, they are incompatible with each other, and as a result, they cannot operate on the same physical links in a network without losing real-time guarantees~\cite{duerr16}.
Consequently, the \IEEE{802.1 \glsfirst{tsn}} task group~\cite{8021tsn} has been working since 2012 on standardizing real-time and safety-critical enhancements for Ethernet.

In Ethernet networks, messages are transmitted between end systems as \emph{frames}.
Frames are forwarded on links, through switches, on a route from sender to one or multiple receivers.
They queue up in switches during transmission while waiting for the next link in the route to become available.
Each switch has multiple queues, and frames are filtered into queues based on their priority.
When a link becomes available a new frame is chosen for transmission starting from the highest priority queue.
Hence, the queueing delay for each frame depends on its priority, on how many other frames are queued in front of it, and on the availability of the next link.
This leads to network congestion causing nondeterministic behavior.

We consider networks as defined in the \gls{tsn} standard.
In such networks, applications with different timeliness requirements coexist. We divide network traffic into three traffic types based on criticality level:
\gls{tt} flows for critical traffic, e.g., hard real-time communication, \gls{avb} flows with bounded end-to-end latency, and finally \gls{be} messages. In this paper, we focus on \gls{tt} traffic. \gls{tsn} defines mechanisms for forwarding queued frames from a specific queue at precise points in time.
This is implemented by blocking the other queues and relies on static schedule tables, denoted \glsfirstplural{gcl}, for deciding when to block each queue.
Along with \IEEE{802.1ASrev} for achieving clock synchronization across network devices, these mechanisms provide the basic building blocks for determinism and bounded end-to-end latency.
It is, however, beyond the scope of \gls{tsn} to define how to synthesize \glsplural{gcl}.

The focus of this paper is on the scheduling problem for \gls{tt} traffic: Given the \gls{tsn} network topology, the \gls{tt} frames and their routing we want to determine the \glsplural{gcl} for \gls{tt} traffic in order to guarantee timeliness.
The schedule must ensure that all deadlines are satisfied and should be optimized to meet industrial application demands~\cite{ousterhout2013, craciunas16combined} for minimal latency and jitter. The schedule must facilitate a high \gls{qos} for \gls{avb} and \gls{be} traffic by using a minimal number of queues in switches, thereby leaving as many queues as possible available for the lower-priority traffic classes. We propose a \gls{grasp}-based strategy for synthesizing \glsplural{gcl} with minimized queue utilization and end-to-end latency.

There is a strong interest in the industry in adaptive networks that can support safety-critical real-time applications~\cite{emc2}. For example, industrial applications require dynamic reconfiguration to meet new business demands, allowing computation and communication services to evolve over time with minimal disruption. Hence, we are interested in solutions which can be used to perform runtime reconfiguration, i.e., they have to derive good quality schedules in a short time. In \gls{tsn}, runtime reconfiguration is supported by the extension \IEEE{802.1Qcc}, which allows a ``Centralized Network Configurator'' to update the network configuration at runtime.

\subsection{Related Work}
Synthesizing schedules for distributed real-time applications is a well-studied problem.
It consists of two coupled problems: Scheduling tasks at the processor level and scheduling frames at the network level.
Real-time applications rely on timeliness at both levels.
In \cite{craciunas16combined} the two problems are solved jointly using a \gls{smt} approach, and Zhang et al.~\cite{zhang2014} propose a \gls{mip} approach for the joint problem.

In this paper, we focus on synthesizing schedules at the network level.
In the context of TTEthernet, numerous strategies have been proposed for scheduling frames on network links.
Steiner~\cite{steiner2010} models the problem as a set of constraints to which a feasible solution is found using an \gls{smt}-solver.
In \cite{steiner2011} the same author proposes an extension to the \gls{smt} approach where blank spaces are left between TT frames to avoid starvation of lower-priority frames, thereby improving \gls{qos}. Pozo et al.~\cite{pozo2015} improve scalability of the \gls{smt}-based approach for TTEthernet by decomposing the problem into subproblems which are solved independently.

Other approaches have also been proposed, such as a Tabu Search-based metaheuristic~\cite{tamas2014} which minimizes end-to-end latency of lower priority traffic. Avni et al.~\cite{avni2016} propose a scheduling strategy where switches have an error-recovery protocol, enabling them to utilize secondary paths in case of link failures.

Recent work has addressed the scheduling problem for \gls{tsn} as well. \gls{tsn} and TTEthernet are similar in many ways, but differ in some significant aspects: Messages in TTEthernet consist of a single frame, whereas TSN messages may consist of multiple frames.
Furthermore, TTEthernet schedule tables are specified for individual frames, whereas \gls{tsn} specifies schedules for the queues, not frames.
Consequently, all frames sharing the same queue are affected by the associated schedule table.
As a result, the work on TTEthernet scheduling is not directly applicable to \gls{tsn} networks.

D{\"u}rr and Nayak~\cite{duerr16} relate the \gls{tsn} scheduling problem to the ``No-wait Job-shop Scheduling Problem'' in order to achieve schedules with minimum network delay for \gls{tt} messages. They assume that a single queue is reserved for TT traffic in every outgoing port and that all messages are transmitted with the same period. With these assumptions they achieve near-optimal schedules with a Tabu Search-based approach.

Craciunas et al.~\cite{craciunas16} identify issues affecting determinism in \gls{tsn} schedules and define additional \gls{tsn}-specific constraints for overcoming these issues.
Furthermore, they propose an \gls{smt}-based approach where the assumptions from \cite{duerr16} have been relaxed, i.e., TT traffic may utilize multiple queues, and messages have individual periods.
Assigning \gls{tt} traffic to multiple queues adds flexibility and enables finding feasible schedules in scenarios with high link utilization.
However, queue usage should in general be minimized where possible in order to make queues available for lower-priority traffic.
Thus, \cite{craciunas16} also presents an \gls{omt} approach for minimizing queue usage.
Pop et al.~\cite{pop16} propose an \gls{ilp}-based strategy as an alternative to the \gls{smt} approach of \cite{craciunas16}, but apart from that they define the problem in the same way.

\textbf{Contribution:} The previous work has proposed SMT and ILP formulations for the \gls{gcl} synthesis problem such that the queue usage is minimized. These solutions do not scale for realistic problem sizes, and do not address the latency minimization, which is important for a large class of applications. In this paper, we formulate the \gls{gcl} synthesis as a multi-objective optimization problem, targeting both queue usage and latency minimization. We propose a scalable \gls{grasp}-based approach, which relies on a scheduling heuristic in order to provide good quality solutions in a short time for large test cases.
\section{Architecture Model} \label{sec:architecture_model}
The architecture model is an abstract representation of the physical \gls{tsn} network, including end systems, switches, and data links.
The topology is modeled as a directed graph \gls{network}, where the vertices \gls{network_v} represent devices in the network, i.e., end systems, denoted \gls{es_set}, and switches, denoted \gls{sw_set}.
Hence, $\gls{network_v} = \gls{es_set} \cup \gls{sw_set}$.
The set of edges \gls{network_e} represents data links.
A directed edge from $v_a \in \gls{network_v}$ to $v_b \in \gls{network_v}$ represents a one-way communication link from $v_a$ to $v_b$.
Thus, a physical full-duplex link between devices $v_a$ and $v_b$ results in two edges, denoted $\l \in \gls{network_e}$ and $\l[v_b,v_a] \in \gls{network_e}$.
\autoref{fig:architecture_model} shows the topology of a network with three end systems, $ES_1$, $ES_2$, $ES_3$, and a single switch, $SW_1$.
\begin{figure}[b]
\centering
  \includegraphics[width=0.60\linewidth]{figures/architecture_model}
\caption{Network topology}
 \label{fig:architecture_model}
\end{figure}

Three attributes, $s$, $d$, and $c$, are associated with each link, \l{}. An object-oriented notation is used to refer to attributes of a specific link, e.g., $\l.s$. Attribute $\l.s$ denotes the transmission rate of \l, typically 100 Mbps or 1 Gbps. $\l.d$ denotes the propagation delay, which is proportional to the length of the physical link.
There is exactly one egress port for every link and, hence, the notation \l{} refers to the link from $v_a$ to $v_b$ as well as the egress port in $v_a$ associated with the link to $v_b$. $\l.c$ refers to the number of queues available in egress port \l.
We assume $\l.c = 1$ if $v_a \in \gls{es_set}$ and $\l.c = 8$ if $v_a \in \gls{sw_set}$.

A route, $r_{k}$, is an ordered sequence of links connecting sending end system $v_a$ with receiving end system $v_b$.
The sequence represents a data path in the network for sending messages from $v_a$ to $v_b$.
Every route starts and ends in end systems with one or multiple intermediate switch vertices.
\autoref{fig:architecture_model} shows two routes: $r_{1} = (\l[ES_1, SW_1],\l[SW_1, ES_3])$, and $r_{2} = (\l[ES_2, SW_1],\l[SW_1, ES_3])$.
Note that there may be multiple redundant routes connecting the same pair of end systems depending on the topology of the network.
We let $R_{a,b}$ denote the set of all routes connecting $v_a$ with $v_b$.
Due to the simplicity of the topology in~\autoref{fig:architecture_model} all pair of end systems have only one route, e.g., $R_{1,3} = \{ r_1 \}$ and $R_{2,3} = \{ r_2 \}$.
\section{Application Model} \label{sec:application_model}
The application model represents \glsfirst{tt} communication on network \gls{network}. In this paper, we consider periodic \gls{tt} messages, called \emph{flows}. The set of \gls{tt} flows is denoted \gls{flows_set}. Associated with each \gls{tt} flow \s{} is the tuple of attributes $(v_s, v_t, r, T, D, P, k)$.
Attribute $v_s$ denotes the sending end system, $v_t$ denotes the receiving end system, and $r \in R_{s,t}$ denotes a route in the network from $v_s$ to $v_t$ on which the message is forwarded. $T$ denotes the period in microseconds after which the message is repeated. $D$ denotes the relative deadline for when the entire message must be received in $v_t$. The deadline upper bounds the end-to-end latency of the flow. We assume that $D \leq T$. $P$ denotes the payload, or data size, of \s, and $k$ denotes the number of frames required to carry the entire message.
Hence, every $T$ microseconds, a message of $P$ bytes, in the form of a sequence of $k$ frames, is sent on the network via route $r$ from end system $v_s$ to end system $v_t$, such that the end-to-end latency is below $D$. \autoref{tab:application_model} shows two sample flows \s[1] and \s[2] with associated attributes for the architecture in \autoref{fig:architecture_model}.

A single Ethernet frame transmits a payload of at most 1500 bytes (B), the so-called \gls{mtu}.
If the data size is larger than \gls{mtu}, the message is fragmented into multiple frames.
The number of frames needed to transmit a message with data size $P$ is $k = \lceil \frac{P}{MTU} \rceil$, where it is assumed that all frames are \gls{mtu}-sized, except for the last one.
\begin{table}[b]
\centering
\begin{tabular}{cccccccc}
flow  & $v_s$  & $v_t$  & $r$   & $T$       & $D$       & $P$                 & $k$ \\ \hline
\s[1] & $ES_1$ & $ES_3$ & $r_1$ & \mus{100} & \mus{100} & 1500 B                 & 1   \\
\s[2] & $ES_2$ & $ES_3$ & $r_2$ & \mus{150} & \mus{150} & 4500 B & 3  
\end{tabular}
\caption{Set of \gls{tt} flows \gls{flows_set} with attributes}
\label{tab:application_model}
%\vspace{-3ex}
\end{table}

A flow \emph{instance}, denoted $\s[i][\l]$, is the instance of a flow \s{} on a particular link \l.
There is one such flow instance for each link in the route $\s.r$.
The flow instance concept is introduced to capture the assignment of frames to egress port queues.
Recall that \l{} denotes the link between devices $v_a$ and $v_b$ as well as the egress port in $v_a$ towards $v_b$.
The frames of $\s[i][\l]$ are queued in egress port \l{} during transmission.
An attribute $\rho$ is associated with each flow instance. It specifies to which queue the frames of \s{} are assigned in egress port \l, thus it is upper bounded by the number of queues in the device ($\l.c$).
The flow instance concept ensures that all frames of \s{} are assigned the same queue in \l, while allowing the frames to be assigned different queues in different egress ports along route $\s.r$.

The queue assignments for all flow instances $\s[i][\l]$ of egress port \l{} implicitly define a division of the $\l.c$ queues into \gls{tt} queues dedicated to \gls{tt} flows, and the remaining queues for other traffic classes. Note that if \l{} is used for critical traffic it must have at least one \gls{tt} queue.

We use \f[i,m][] to denote the $m$th frame of flow \s{}. Analogous to flow instance, \emph{frame instance} \f{} denotes the transmission of the $m$th frame of flow \s{} on link \l{}. \f{} is associated with the attribute pair $(L, \phi)$, where $L$ denotes the transmission duration in microseconds on \l{}, and $\phi$ denotes the microsecond offset of \f{} within period $\s.T$, We restrict that $\phi \in [0, \s.T-L]$, to ensure that all frames are fully transmitted within the period.

The transmission duration $\f.L$ is based on the transmission rate and propagation delay of the physical link, and the total number of bytes in the packet.
Suppose that the frame has a payload of \gls{mtu} (1500 bytes), resulting in a packet size of 1542 bytes, then the transmission duration on a 1 Gbps link with negligible propagation delay is calculated in \autoref{eq:frame_duration}.
\begin{equation}
   \label{eq:frame_duration}
   \f.L = \frac{1542 \text{ B}}{[v_a,v_b].s} + [v_a, v_b].d = \frac{1542 \text{ B}}{1 \text{ Gbps}} =  \mus{12.336}
\end{equation}
The offset $\f.\phi$ defines the start time for transmission of \f{} within its period $\s.T$. The frame is repeatedly sent at the times given in \autoref{eq:frame_offset}.
\begin{equation}
   \label{eq:frame_offset}
   \phi, \quad \s.T + \phi , \quad 2 \cdot \s.T + \phi, \quad 3 \cdot \s.T + \phi, \quad \ldots
\end{equation}
\section{TSN Protocol} \label{sec:tsn}
\gls{tsn} is based on a switched multi-hop network architecture adopted from \IEEE{802.3 Ethernet}.
Switches interconnect end systems via \emph{full-duplex} links, meaning that the physical links enable transmission in both directions simultaneously.
\autoref{fig:architecture_full_detail} shows a \gls{tsn} network architecture corresponding to the architecture model in \autoref{fig:architecture_model}.

\begin{figure}[b]
    \centering
    \includegraphics[width=0.40\textwidth]{figures/architecture_full_detail}
    \caption{\gls{tsn} network with internal queues and gates}
    \label{fig:architecture_full_detail}
%    \vspace{-0.5cm}
\end{figure}

Ethernet frames contain \IEEE{802.1Q} headers~\cite{8021q}, with two fields of importance to \gls{tt} traffic:
\begin{enumerate*}[(1)]
   \item \gls{vid} is a 12-bit field specifying the Virtual LAN of a frame. This is used to distinguish frames from different messages.
   \item \gls{pcp} is a 3-bit field specifying the priority level, i.e., the traffic class such as \gls{tt}, \gls{avb}, or \gls{be}.
   Furthermore, it defines which queue the frame is assigned to within a switch.
\end{enumerate*}

An Ethernet switch has ingress (incoming) and egress (outgoing) ports connecting it via links to surrounding switches and end systems.
Each egress port typically has eight queues for storing frames that wait to be forwarded on the corresponding link, as shown for $SW_1$ in \autoref{fig:architecture_full_detail}.
When frames from critical flows arrive at ingress ports they are filtered into queues based on their stream identification using the per stream filtering and policing functionality defined in \IEEE{802.1Qci}.

\IEEE{802.1ASrev} defines a network-wide time-synchronization protocol which effectively achieves a global notion of time across all switches and end systems.
Clocks must be comparable across devices, if forwarding of frames is to be controlled accurately.
The time-synchronization protocol achieves a microsecond network precision, denoted by $\delta$.
The difference in the internal clocks of any two devices at any point in time is at most $\delta$.
\begin{figure*}[t]
\centering
	\begin{subfigure}{0.4\linewidth}
      \centering
      %\small
      \bgroup
      \setlength\tabcolsep{0.1cm}
      \begin{tabular}{lccccccccccccc}
             & $t_0$ & $t_1$ & $t_2$ & $t_3$ & $t_4$ & $t_5$ & $t_6$ & $t_7$ & $t_8$ & $t_9$ & $t_{10}$ & $t_{11}$ & $t_{12}$ \\ \hline
      $q_1$  & 0     & 1     & 0     & 0     & 0     & 0     & 1     & 0     & 0     & 0     & 1        & 0        & 0        \\
      $q_2$  & 0     & 0     & 1     & 0     & 1     & 0     & 0     & 0     & 1     & 0     & 0        & 1        & 0        \\
      $q_3$  & 1     & 0     & 0     & 1     & 0     & 1     & 0     & 1     & 0     & 1     & 0        & 0        & 1        \\
      \ldots &      &      &      &      &      &      & \hspace{-0.2cm} \ldots \hspace{-0.2cm}     &      &      &      &         &         &         \\
      $q_8$  & 1     & 0     & 0     & 1     & 0     & 1     & 0     & 1     & 0     & 1     & 0        & 0        & 1       
      \end{tabular}
      \egroup
      \caption{\gls{gcl} for egress port \l[SW_1, ES_3]}
      \label{tab:gcl_sample}
	\end{subfigure}
	\begin{subfigure}{0.55\linewidth}
      \centering
      \includegraphics[width=\textwidth]{figures/example_schedule}
	   \caption{Example schedule}
      \label{fig:sample_schedule}
	\end{subfigure}
   \caption{Example \gls{gcl} and schedule; \autoref{tab:gcl_sample} corresponds to Row \l[SW_1, ES_3] in \autoref{fig:sample_schedule}.}
   \label{fig:sample}
\end{figure*}
Utilizing the global network clock, \IEEE{802.1Qbv} defines a \gls{tas} concept which enables scheduling of high priority traffic.
\gls{tas} controls a gate for each queue.
Frames in a queue are only eligible for transmission if the queue gate is open.
$SW_1$ in \autoref{fig:architecture_full_detail} has two ingress ports and a single egress port.
Egress ports are depicted as boxed arrow tails, ingress ports as boxed arrow heads, and data links as the lines in-between.
Every queue of the egress port has an associated gate.
In our example in \autoref{fig:architecture_full_detail}, the gate of the topmost queue is open, enabling the red frame to be selected for transmission.
Interference from lower priority traffic is prevented by closing the gates of the remaining queues.
When the egress port is idle, the next frame is selected for transmission from the queue with highest priority among the queues with open gates.
Opening queues in a mutually-exclusive fashion, allows for full control of frame forwarding.

A \glsfirst{gcl} defines for each egress port, when the queue gates are open and closed.
In \autoref{fig:architecture_full_detail} they are depicted as white tables below queues. 1 and 0 in the \gls{gcl} represent an open and closed gate, respectively.
For instance, the gate in $ES_1$ is open at time $t_0$ but closed at $t_1$ and $t_2$, and conversely for the gate in $ES_2$.
Using the \glsplural{gcl} to schedule forwarding of frames in a route from sender to receiver, enables \gls{tt} traffic suitable for hard real-time communication. \autoref{tab:gcl_sample} shows the full \gls{gcl} for all time events of egress port $\l[SW_1,ES_3]$.

The \glsplural{gcl} can be constructed in such a way that \gls{avb} and \gls{be} traffic are prevented from initiating transmission in time slots reserved for \gls{tt} frames.
However, nondeterminism could still occur due to interference with other \gls{tt} flows. When a frame is scheduled for transmission on a link in a given time interval, the corresponding \gls{gcl} is set to open the associated gate in that interval.
Suppose something goes wrong, so the frame is not fully received, or is not the first frame in the queue as expected. Then the link transmits the wrong frame or remains idle when it should be transmitting.
Consequently, nondeterminism is introduced, which means timeliness is compromised.
Nondeterminism can occur in queues in two scenarios:
\begin{enumerate*}[(i)]
   \item Frames arriving on different ingress ports are scheduled to arrive at roughly the same time.
   Due to the clock synchronization error, the order in which frames are queued is nondeterministic, and could vary from period to period.  \autoref{fig:non_determinism_arrival_times} illustrates nondeterminism due to clock synchronization error.
   
   \item If a frame is lost for some reason, it of course introduces nondeterminism for that particular flow, but it could also affect frames of other flows if they are in the same queue, as shown in \autoref{fig:non_determinism_frame_loss}.
\end{enumerate*}
To enforce determinism, queue-sharing flows should be scheduled carefully.
Frames of such flows must be scheduled so their arrival times are so far apart that the arrival order is deterministic even with the maximal clock synchronization error ($\delta$), and such that only frames of one of the flows are present in the queue at a time.
This is shown in \autoref{fig:non_determinism_solution}.

\begin{figure}[t]
	\centering
	\begin{subfigure}{\linewidth}
      \centering
	   \includegraphics[width=0.49\linewidth]{figures/non_determinism_arrival_times_a}
      \includegraphics[width=0.49\linewidth]{figures/non_determinism_arrival_times_b}
	   \subcaption{Clock synchronization error}
      \label{fig:non_determinism_arrival_times}
	\end{subfigure}
	\begin{subfigure}{\linewidth}
      \centering
	   \includegraphics[width=0.49\linewidth]{figures/non_determinism_frame_loss_a}
	   \includegraphics[width=0.49\linewidth]{figures/non_determinism_frame_loss_b}
	   \subcaption{Frame loss}
      \label{fig:non_determinism_frame_loss}
	\end{subfigure}
	\begin{subfigure}{\linewidth}
      \centering
	   \includegraphics[width=0.49\linewidth]{figures/non_determinism_solution_a}
	   \includegraphics[width=0.49\linewidth]{figures/non_determinism_solution_b}
	   \subcaption{Determinism is enforced by scheduling flows far apart}
      \label{fig:non_determinism_solution}
	\end{subfigure}
   \caption{Non-deterministic queue behaviour}
	\label{fig:non_determinism}
\end{figure}


%a lower priority frame is already transmitting on the link at the beginning of a time slot.
%There are two ways to prevent this. The first option is to place an MTU-sized \emph{guard band} before every \gls{tt} time slot.
%The guard band closes all other queues well in advance to ensure that the link is available when it is time to transmit the \gls{tt} frame.
%This option is undesirable because it decreases the bandwidth available for lower-priority traffic.
%The second option is to let the \gls{tt} frames \emph{preempt} to lower priority queues as defined in \IEEE{802.1Qbu}.
%When a frame is preempted, its transmission is temporarily paused, so the link becomes available to a higher-priority frame.
%Once transmission of the higher-priority frame completes, transmission of the preempted frame is resumed. A header is added to each fragment of the preempted frame, which increases the total overhead associated with the frame.
%However, the overhead is negligible compared to an MTU-size guard band. Thus, preemption is the preferred method, if supported by the hardware.

Each \gls{gcl} has a temporal granularity which we refer to as the \emph{macrotick}.
For simplicity, and without loss of generality, we assume that all \gls{gcl}s have a macrotick of \mus{1}.
This makes comparing \glsplural{gcl} across different switches straightforward.

\autoref{fig:architecture_full_detail} shows the state of the network at time $t_1$. At this point in time, the first gate in $SW_1$ is open, while the others are closed, as defined in column $t_1$ of the \gls{gcl}.
A red frame is queued in $q_1$ of $SW_1$, and the link connecting $SW_1$ and $ES_3$ is idle.
Consequently, the frame in $q_1$ is chosen for transmission because $q_1$ is the only queue with an open gate.
This is illustrated with the red arrow.
In parallel, a frame, illustrated with a blue arrow, is being transmitted from $ES_2$ to $SW_1$ because the gate in $ES_2$ is open.
The PCP field of the frame dictates that it is filtered into $q_2$ in $SW_1$.
The gate of this queue is closed at time $t_1$, so it cannot interfere with transmission of the red frame.
At time $t_2$, the red frame has finished transmission.
From the $t_2$-column of the \gls{gcl} in $SW_1$, it is seen that $q_1$ closes and $q_2$ opens, causing the blue frame to be transmitted to $ES_3$.

\section{GCL Schedule} \label{sec:gcl_schedule}
\glsplural{gcl} for all egress ports make up a deterministic schedule of when to send \gls{tt} frames on links.
Because of the periodic nature of flows, the \glsplural{gcl} have a certain cycle time after which they start over from the beginning.
Consequently, the corresponding schedule repeats after this cycle duration which we denote the \emph{hyperperiod}~\cite{buttazzo1997}.
The hyperperiod depends on the periods of the individual flows because the period of each flow must be a divisor of the hyperperiod, i.e., every flow period repeats an integral number of times within the hyperperiod.
Hence, the hyperperiod is the \gls{lcm} of all the flow periods.

\autoref{fig:sample_schedule} shows a schedule for transmitting the frames of \s[1] and \s[2] from \autoref{tab:application_model} illustrated as a variant of a Gantt chart~\cite{sinnen2007}. The schedule uses \autoref{tab:gcl_sample} as GCL for egress port \l[SW_1, ES_3].
The horizontal axis represents the time dimension, that is, when frames are transmitted. The vertical axis represents links and queues.
A box represents the transmission of a frame on a link. For instance, the box labeled ``1.1'' on the row for \l[ES_1, SW_1] represents transmission of \f[1,1][{\l[ES_1, SW_1]}].
In the figure, \s[1] is assigned to the first queue of \l[SW_1, ES_3], and \s[2] to the second, corresponding to $\s[1][{\l[SW_1, ES_3]}].\rho = 1$ and $\s[2][{\l[SW_1, ES_3]}].\rho = 2$.
The thin rows below link \l[SW_1, ES_3] illustrate when frames are in the queues of egress port \l[SW_1, ES_3]. A frame is in a queue from the time transmission is initiated on the previous link, until the time transmission is initiated on the egress port associated with the queue. Hence, in \autoref{fig:sample_schedule}, \f[1,1][] is in $q_1$ of egress port \l[SW_1, ES_3] from $t_0$ until $t_1$ where transmission of \f[1,1][{\l[SW_1, ES_3]}] is initiated.

Flow \s[1], with period \mus{100}, has three repetitions within the hyperperiod of \mus{300}, while \s[2], with period \mus{150}, has two repetitions.
Notice, how the start time of a frame is at the same offset in each period.
For instance, $\f[1,1][{\l[ES_1, SW_1]}]$ is transmitted at times \mus{0}, \mus{100}, and \mus{200}, corresponding to $\f[1,1][{\l[ES_1, SW_1]}].\phi = \mus{0}$. The model does not allow for different $\phi$ values in different repetitions.

There is an equivalence between the set of \glsplural{gcl} for all egress ports and a schedule like the one presented in \autoref{fig:sample_schedule}.
That is, a set of \glsplural{gcl} can be constructed from a schedule and vice versa.
\section{Problem Formulation} \label{sec:problem_formulation}
The problem addressed in this paper is: Given a \gls{tsn} network topology \gls{network}, and a set of \gls{tt} flows \gls{flows_set} and their routing, map flows to egress port queues and determine \glsfirstplural{gcl} for all egress ports in the network. This corresponds to finding a feasible schedule, i.e., feasible assignments for the following two sets of variables:
\begin{enumerate*}[(i)]
   \item A queue $\s[i][\l].\rho$ in egress port \l{} for each flow instance.
   \item An offset $\f.\phi$ for the periodic transmission on data link \l{} for each frame.
\end{enumerate*}

A feasible schedule satisfies all hardware-imposed constraints while meeting safety-critical timeliness requirements.
We wish to determine a solution such that two objectives are minimized:
\begin{enumerate*}[(i)]
   \item The number of queues used by \gls{tt} flows, and
   \item the end-to-end latency for each flow.
\end{enumerate*}
The reason, for minimizing the number of \gls{tt} queues, is to maximize the number of queues that remain available for lower-priority traffic, such as \gls{avb} and \gls{be}.
End-to-end latency should be minimized because it is an important \glsfirst{qos} metric in most applications.
Furthermore, a minimized end-to-end latency means frames spend less time waiting in queues, which helps reduce memory requirements for queues.

%\subsection{Schedule Feasibility} \label{sec:schedule_feasibility}
A schedule must satisfy certain constraints, defining the solution space of feasible schedules, i.e. schedules that are both realizable in physical \gls{tsn} networks and meet the timing requirements. The three most essential constraints are:

\begin{description}[leftmargin=0cm]
   \item[Link congestion.] A data link is limited by its hardware to only transmit a single frame at a time, i.e., frames on the same link cannot overlap in the time domain. This corresponds to the property that boxes on the same row of \autoref{fig:sample_schedule} do not overlap. The link can be seen as a critical section, that can only be occupied by a single frame at a time.
   \item[Flow transmission.] A switch cannot forward a frame until the entire frame has been buffered in the switch.
This introduces a forwarding delay for each hop from source to destination. Due to the small synchronization error of the clocks between devices, the exact time when the entire frame has been received in a particular switch is unknown.
Consequently, the time for forwarding the frame on the next link should take into account the worst-case synchronization error, $\delta$.
   \item[Bounded end-to-end latency.] All \gls{tt} flows must arrive within their relative deadline, i.e., the end-to-end latency cannot exceed the deadline. End-to-end latency is defined as the time from the sender initiates transmission of the first frame and until the last frame arrives at the receiver.
   \item[Deterministic queues.] Analogously to the \emph{link congestion} property, a queue can be considered a critical section, which can only be occupied by frames from a single flow at a time. In \autoref{fig:sample_schedule} this corresponds to the property that queue utilization boxes do not overlap in the time domain, if they belong to the same queue. In addition, there should be a $\delta$-sized spacing between queue utilization boxes of frames arriving at different ingress ports, to account for a worst-case synchronization error as explained in \autoref{sec:tsn}.
\end{description}

\subsection{Motivational Example}
The example from \autoref{fig:sample_schedule} serves as a motivational example. The schedule is feasible with respect to the constraints of \autoref{sec:problem_formulation}, but is not optimized with respect to any of the two criteria: Queue usage and end-to-end latency. In order to improve queue usage, the frames of \s[1] and \s[2] should be rearranged in such a way that they both share the same queue in \l[SW_1, ES_3] without occupying the queue at the same time. \autoref{fig:example_schedule_queue} shows such a schedule, where they both use $q_1$. Notice that the queue utilization boxes do not overlap.
\begin{figure}[t]
\centering
	\begin{subfigure}{\linewidth}
        \centering
        \includegraphics[width=\linewidth]{figures/example_schedule_queue}
        \caption{Minimum queue usage}
        \label{fig:example_schedule_queue}
   \end{subfigure}
	\begin{subfigure}{\linewidth}
        \centering
        \includegraphics[width=\linewidth]{figures/example_schedule_e2e}
        \caption{Minimum end-to-end latency}
        \label{fig:example_schedule_e2e}
   \end{subfigure}
   \caption{Improved alternatives to \autoref{fig:sample_schedule}}
\end{figure}


Minimizing queue usage has a negative effect on end-to-end latency. In \autoref{fig:example_schedule_queue} the frames of \s[2] have been spaced further apart, thereby increasing the end-to-end latency. Instead, the schedule could be optimized with respect to end-to-end latency as shown in \autoref{fig:example_schedule_e2e}.
In this schedule, two queues are used but the frames of \s[2] are grouped closer together compared to \autoref{fig:sample_schedule} and \autoref{fig:example_schedule_queue} resulting in a smaller end-to-end latency. This motivational example shows that a desirable schedule is a tradeoff between queue usage and end-to-end latency.
\section{GRASP-based GCL Synthesis}
The NP-complete flow-shop scheduling problem~\cite{garey1976} reduces to the \gls{tt} scheduling problem~\cite{raagaardthesis}.
Hence, we solve the NP-hard problem of scheduling \gls{tt} flows using \glsfirst{grasp}~\cite{resende14}.
The \gls{grasp} metaheuristic is well-suited for combinatorial optimization problems, where an initial solution is efficiently constructed in a greedy manner.
Each iteration of \gls{grasp} consists of two phases: A construction phase, where an initial feasible solution is built, and a search phase, where a neighborhood around the initial solution is examined for improving solutions. The construction phase contributes with diversification, and the local search with intensification, ensuring convergence towards a global optimum. Note, however, that \gls{grasp}, being a metaheuristic, is not guaranteed to find the optimum.

For the \gls{tt} scheduling problem, the construction phase is a polynomial-time algorithm, which greedily schedules flows one at a time, thereby constructing an initial feasible solution. The local search phase looks for improving solutions by removing parts of the schedule and rescheduling them differently.

\autoref{alg:grasp} describes the overall strategy of \gls{grasp} for the \gls{tt} scheduling problem.
\begin{@empty}
\begin{algorithm}[b]
\small
\caption{GRASP metaheuristic} \label{alg:grasp}
	\begin{algorithmic}[1]
		\Function{GRASP}{\gls{flows_set}, $\gamma$, $\pi$, \gls{heuristics_set}}
			\State $x \gets \emptyset$ \label{algline:x_empty}
			\Repeat
				\State $x' \gets \Call{GreedyRandomized}{\gls{flows_set}, \gamma, \gls{heuristics_set}}$ \label{algline:construction_phase}
				\State $x' \gets \Call{LocalSearch}{x', \gls{flows_set}, \pi, \gls{heuristics_set}}$ \label{algline:local_search_phase}
				\If{$f(x') < f(x)$} \label{algline:objective_comparison}
					\State $x \gets x'$ \label{algline:update_best}
				\EndIf
			\Until{exceeding time limit} \label{algline:stop_criterion}
			\State \Return{$x$}
		\EndFunction
	\end{algorithmic}
\end{algorithm}
\end{@empty}
As input it takes the problem instance, i.e., the set of flows \gls{flows_set}, as well as two parameters, $\gamma$ and $\pi$, related to the construction and local search phases, respectively. Furthermore, it takes a set of heuristics \gls{heuristics_set} for scheduling individual flows. In \autoref{sec:heuristics} we discuss the details of the heuristics.
\autoref{alg:grasp} outputs a feasible schedule $x$, which is the best solution found throughout the search. Initially, $x$ is empty (line \ref{algline:x_empty}), indicating that a feasible solution is yet to been found, i.e., the set of scheduled flows is empty.

In each iteration, a new solution $x'$ is generated in a greedy randomized fashion (line \ref{algline:construction_phase}) using a constructive heuristic. The heuristic contains a random element to ensure that different parts of the solutions space are explored in each iteration. The parameter $\gamma$ defines the level of randomness. Too much randomness affects the quality of the initial schedules, whereas too little randomness affects diversification. The construction phase is elaborated in \autoref{sec:construction_phase}.

The initial solution is subsequently optimized via a local neighborhood search until reaching a local optimum (line \ref{algline:local_search_phase}). The local search destroys and repairs the current schedule to obtain new solutions. The parameter $\pi$ specifies how much to destroy/repair in each iteration of the local search. If a new solution results in a better solution than the current best known, then the best solution is updated (lines \ref{algline:objective_comparison} and \ref{algline:update_best}). This repeats until a given execution time limit has been reached. The local search phase is described in \autoref{sec:local_search_phase}.

\subsection{Objective Function} \label{sec:objective_function}
%The quality of schedules is evaluated based on an objective function $f$, which maps schedules to real numbers, i.e., $f: x \to \mathbb{R}$. The function is defined as a weighted sum of the two optimization metrics: Queue usage, denoted $\Kappa$, and end-to-end latency, denoted $\Lambda$, as shown in \autoref{eq:f_objective}.
%\begin{equation}
%   \label{eq:f_objective}
%   f(x) = c_1 \cdot \Kappa(x) + c_2 \cdot \Lambda(x)
%\end{equation}
%$\Kappa(x)$ denotes the total number of queues dedicated to \gls{tt} flows in $x$, i.e., the sum of \gls{tt} queues over all switches in \gls{sw_set}. $\Lambda(x)$ denotes the total end-to-end latency, i.e., the sum of individual end-to-end latencies of all flows in $x$. As a result, the weights $c_1$ and $c_2$ specify the trade-off between queue usage and end-to-end latency. The weights should be set by the systems engineer to reflect desired schedules.

%\michael{Rewritten to address reviewers concern regarding normalization:}
The quality of schedules is evaluated based on an objective function $f$, which maps schedules to real numbers, i.e., $f: x \to \mathbb{R}$.
\begin{modifiedparagraph}
The function is defined as a weighted sum of the two optimization metrics: The normalized queue usage, denoted $\Kappa_N$, and the normalized end-to-end latency, denoted $\Lambda_N$, as shown in \autoref{eq:f_objective}.
\begin{equation}
   \label{eq:f_objective}
   f(x) = c_1 \cdot \Kappa_N(x) + c_2 \cdot \Lambda_N(x)
\end{equation}
$\Kappa_N(x)$ is a mapping of queue usage for \gls{tt} traffic to the interval $[0;1]$ as shown in \autoref{eq:f_normalization}.
\begin{equation}
   \label{eq:f_normalization}
   K_N(x) = \frac{ K(x) - \lbound{K}(x)}{\ubound{K}(x) - \lbound{K}(x)}
\end{equation}
where $K(x)$ denotes the total number of queues used for \gls{tt} traffic across all egress ports. \lbound{K}(x) and \ubound{K}(x) are lower and upper bounds, respectively. The lower bound is assuming a single \gls{tt} queue in all egress ports forwarding \gls{tt} traffic, and the upper bound assumes the minimum of the available queues in the egress port and the number of \gls{tt} flows forwarded through that egress port.

The total end-to-end latency is normalized in $\Lambda_N$ in a similar way. The lower bound assumes that all flows are scheduled independently, i.e., without interference from other \gls{tt} flows. The upper bound assumes the worst-case scenario where all flows have end-to-end latencies equal to their deadline. The weights $c_1$ and $c_2$ are set by the system engineers based on their desired trade-off between queue usage and end-to-end latency. 
\end{modifiedparagraph}

%\michael{I think this would be the most fair way to normalize. It is not perfect, though, as they may not be equally pessimistic --- as we have discussed.}

\subsection{Greedy Randomized Heuristic} \label{sec:construction_phase}
The construction phase is a polynomial-time greedy randomized heuristic designed to find feasible schedules which serve as good starting points for the subsequent local search. Hence, it should be computed efficiently, should not produce the same schedule in each iteration, and should not make obvious suboptimal decisions which the local search has to spend much time rectifying.

\autoref{alg:construction_phase} shows the construction phase.
\begin{algorithm}[t] \label{alg:construction_phase}
\small
\caption{Construction phase of GRASP} \label{alg:construction_phase}
   \begin{algorithmic}[1]
      \Function{GreedyRandomized}{\gls{flows_set}, $\gamma$, \gls{heuristics_set}}
         \State $x \gets \emptyset$
         \State $\gls{flows_set}' \gets \Call{SortByPeriod}{\gls{flows_set}}$ \label{algline:heur_order}
         \For{$\s \in \gls{flows_set}'$} \label{algline:for_flow}
            \State $RCL \gets \Call{RestrictedCandidateList}{\gamma, f}$ \label{algline:rcl}
            \For{$h \in \gls{heuristics_set}$}
               \If{$\Call{ScheduleFlow}{x, \s, h} = \textbf{true}$} \label{algline:if_schedule_flow}
                  \State $x' \gets x \cup \s$
                  \State $RCL.\Call{AddCandidate}{x'}$
               \EndIf \label{algline:endif_schedule_flow}
            \EndFor
            \If{$RCL.\Call{Length}{} > 0$} \label{algline:if_rcl_not_empty}
               \State $x \gets RCL.\Call{GetRandomCandidate}$ \label{algline:random_from_rcl}
            \EndIf \label{algline:endif_rcl_not_empty}
         \EndFor \label{algline:end_for_flow}
         \State \Return{$x$}
      \EndFunction
   \end{algorithmic}
\end{algorithm}
Flows are ordered by their period (line \ref{algline:heur_order}). To break ties, the route length is used as an indicator of how difficult flows are to schedule. Flows are scheduled one at a time in this order (lines \ref{algline:for_flow}-\ref{algline:end_for_flow}). Given the current solution $x$, each flow is attempted scheduled using all of the different heuristics in \gls{heuristics_set}. The \gls{rcl} (line \ref{algline:rcl}) is a data structure that keeps track of the $\gamma$ best schedules produced by the heuristics with respect to the objective function $f$. When all heuristics have been considered, a random schedule is chosen from among those in \gls{rcl} (line \ref{algline:random_from_rcl}).

It may be the case that a heuristic in \gls{heuristics_set} is unable to schedule a particular flow. In that case, no candidate is added to \gls{rcl} (lines \ref{algline:if_schedule_flow}-\ref{algline:endif_schedule_flow}). If all heuristics fail, \gls{rcl} is empty (lines \ref{algline:if_rcl_not_empty}-\ref{algline:endif_rcl_not_empty}), i.e., that particular flow is not included in the schedule, making it incomplete. The inner workings of $\Call{ScheduleFlow}{}$ are explained in \autoref{sec:heuristics}.

\subsection{Local Search} \label{sec:local_search_phase}
The purpose of the local search phase is to intensify the search by investigating a well-defined neighborhood of solutions similar to the current solution. This corresponds to schedules where the majority of flows are scheduled exactly as in the current solution. It is likely that a better solution arises from rescheduling only a couple of flows. The local search attempts to identify such rearrangements by removing a small subset of flows, and rescheduling them in a different way.

The neighborhood is defined as follows: All the schedules which can be constructed by removing up to $\pi$ flows from $x$, and subsequently rescheduling them using one of the heuristics in \gls{heuristics_set}. If a new, improving solution is discovered, the neighborhood search is repeated for the new solution. The local search continues in this way until reaching a local minimum, from which no solution from the neighborhood improves the current solution, or until exceeding the time limit.

The destroy and repair mechanisms of the local search rearranges flows compared to the original static order given by $\Call{SortByPeriod}{}$. Thus, the local search can recover from a suboptimal ordering of flows in the construction phase.

\subsection{Scheduling Heuristics} \label{sec:heuristics}
Given an existing partial solution $x$, several feasible schedules exist for a flow \s{}. In this section we present a heuristic approach for scheduling the individual frames of a single flow, while minimizing queue usage. The achieved schedule can then, in turn, be post-processed to minimize end-to-end latency. The heuristic strategy is generalized into multiple variations denoted $\gls{heuristics_set}$.

The heuristic schedules the frames of \s{} sequentially, scheduling each frame on all links before moving on to the next frame. It continues in this way until either all frames in \s{} are successfully scheduled, or until failing to schedule a particular frame, i.e., failing to determine offsets for the frame such that all constraints of \autoref{sec:problem_formulation} are satisfied.

To ensure that the schedule remains feasible after scheduling \s{}, each frame \f[i,m][] must be scheduled such that:
\begin{enumerate}[(i)]
   \item Links are idle when \f[i,m][] is scheduled for transmission, i.e., link \l{} is idle in the interval given in \autoref{eq:link_idle}. 
   \begin{equation} \label{eq:link_idle}
   \left[\f[i,m].\phi; \f[i,m].\phi + \f[i,m].L\right]
   \end{equation} \label{item:link_congestion}
   \vspace{-0.2cm}
   
   \item When \f[i,m][] enters a queue no other flows in $x$ are already in the queue, and \f[i,m][] leaves before other flows in $x$ enter the queue. On two adjacent links \l{} and \l[v_b,v_c], it means that queue $\s[i][{\l[v_b,v_c]}].\rho$ is empty in the interval given in \autoref{eq:empty_queue}.
   \begin{equation} \label{eq:empty_queue}
      \left[\f[i,m][\l].\phi; \f[i,m][{\l[v_b,v_c]}].\phi \right]
   \end{equation} \label{item:deterministic_queues}
   \vspace{-0.2cm}
   
   \item Frame \f[i,m][] is fully received in a switch before being forwarded on the next link, even with a synchronization error of $\delta$, i.e., the inequality in \autoref{eq:flow_tranmission} must be satisfied.
   \begin{equation} \label{eq:flow_tranmission}
      \f[i,m][\l] + \f[i,m].L + \delta \leq \f[i,m][{\l[v_b,v_c]}].\phi
   \end{equation} \label{item:flow_transmission}
   \vspace{-0.2cm}
   
\end{enumerate}
These conditions must hold for all repetitions of \s{} within the hyperperiod. Hence, the partial schedule $x$ is folded into a single period of \s{}, i.e., the schedule is divided into $\s{}.T$-sized segments which are merged into a single $\s{}.T$-sized schedule.
\autoref{fig:heuristic} visualizes the intersected intervals (white boxes) where links are idle and queues are empty. The $\delta$-sized spacing in $q_1$ between queue utilization boxes is introduced to model the worst-case where the clocks have drifted $\delta$~\mus{} apart. 
\begin{figure}[t]
\centering
	\begin{subfigure}{\linewidth}
        \centering
        \includegraphics[width=0.8\linewidth]{figures/heuristic_q1}
        \vspace{-0.2cm}
        \caption{Failed attempt to schedule \s[2] on $q_1$}
        \vspace{0.4cm}
        \label{fig:heuristic_q1}
   \end{subfigure}
	\begin{subfigure}{\linewidth}
        \centering
        \includegraphics[width=0.8\linewidth]{figures/heuristic_q2}
        \vspace{-0.2cm}
        \caption{Feasible schedule for \s[2] on $q_2$}
        %\vspace{0.4cm}
        \label{fig:heuristic_q2}
   \end{subfigure}
   \caption{Scheduling based on link and queue availability} \label{fig:heuristic}
\end{figure}

Condition \ref{item:link_congestion} corresponds to arranging \f[i,m][] within idle-link blocks on each link. Condition \ref{item:deterministic_queues} corresponds to choosing offsets $\f[i,m][\l].\phi$ and $\f[i,m][{\l[v_b, v_c]}].\phi$ such that they are both in the same empty-queue block.

\begin{figure*}[t]
	\centering
   % ASAP
	\begin{subfigure}{0.30\textwidth}
      \centering
	   \includegraphics[width=\linewidth]{figures/scheduling_method_asap}
      \vspace{-0.6cm}
	   \caption{\glsname{asap}}
      \vspace{0.2cm}
      \label{fig:asap}
	\end{subfigure}
	\begin{subfigure}{0.30\textwidth}
      \centering
	   \includegraphics[width=\linewidth]{figures/scheduling_method_asap_r}
      \vspace{-0.6cm}
	   \caption{ASAP-L}
      \vspace{0.2cm}
      \label{fig:asap_l}
	\end{subfigure}
	\begin{subfigure}{0.30\textwidth}
      \centering
	   \includegraphics[width=\linewidth]{figures/scheduling_method_asap_rl}
      \vspace{-0.6cm}
	   \caption{ASAP-LF}
      \vspace{0.2cm}
      \label{fig:asap_lf}
	\end{subfigure}
	\begin{subfigure}{0.30\textwidth}
      \centering
	   \includegraphics[width=\linewidth]{figures/scheduling_method_asapm}
      \vspace{-0.6cm}
	   \caption{ASAPQ}
      \vspace{0.2cm}
      \label{fig:asapq}
	\end{subfigure}
	\begin{subfigure}{0.30\textwidth}
      \centering
	   \includegraphics[width=\linewidth]{figures/scheduling_method_asapm_r}
      \vspace{-0.6cm}
	   \caption{ASAPQ-L}
      \vspace{0.2cm}
      \label{fig:asapq_l}
	\end{subfigure}
	\begin{subfigure}{0.30\textwidth}
      \centering
	   \includegraphics[width=\linewidth]{figures/scheduling_method_asapm_rl}
      \vspace{-0.6cm}
	   \caption{ASAPQ-LF}
      \vspace{0.2cm}
      \label{fig:asapq_lf}
	\end{subfigure}

   % ALAP
	\begin{subfigure}{0.30\textwidth}
      \centering
	   \includegraphics[width=\linewidth]{figures/scheduling_method_alap}
      \vspace{-0.6cm}
	   \caption{\glsname{alap}}
      \vspace{0.2cm}
      \label{fig:alap}
	\end{subfigure}
	\begin{subfigure}{0.30\textwidth}
      \centering
	   \includegraphics[width=\linewidth]{figures/scheduling_method_alap_l}
      \vspace{-0.6cm}
	   \caption{ALAP-F}
      \vspace{0.2cm}
      \label{fig:alap_l}
	\end{subfigure}
	\begin{subfigure}{0.30\textwidth}
      \centering
	   \includegraphics[width=\linewidth]{figures/scheduling_method_alap_lr}
      \vspace{-0.6cm}
	   \caption{ALAP-FL}
      \vspace{0.2cm}
      \label{fig:alap_lf}
	\end{subfigure}
	\begin{subfigure}{0.30\textwidth}
      \centering
	   \includegraphics[width=\linewidth]{figures/scheduling_method_alapm}
      \vspace{-0.6cm}
	   \caption{ALAPQ}
      \label{fig:alapq}
	\end{subfigure}
	\begin{subfigure}{00.30\textwidth}
      \centering
	   \includegraphics[width=\linewidth]{figures/scheduling_method_alapm_l}
      \vspace{-0.6cm}
	   \caption{ALAPQ-F}
      \label{fig:alapq_l}
	\end{subfigure}
	\begin{subfigure}{0.30\textwidth}
      \centering
	   \includegraphics[width=\linewidth]{figures/scheduling_method_alapm_lr}
      \vspace{-0.6cm}
	   \caption{ALAPQ-FL}
      \label{fig:alapq_lf}
	\end{subfigure}
	\caption{Heuristic variations for scheduling frames of a flow in an existing schedule}
	\label{fig:heuristic_variations}
\vspace{-2ex}
\end{figure*}


\autoref{fig:heuristic} illustrates an \gls{asap} approach to scheduling the frames in \s[2]. In \autoref{fig:heuristic_q1}, \s[2] is assigned to $q_1$, whereas in \autoref{fig:heuristic_q2} it is assigned to $q_2$.
%($\s[2][{\l[SW_1, ES_3]}].\rho = 1$ and $\s[2][{\l[SW_1, ES_3]}].\rho = 2$, respectively).
A frame is scheduled on its route, in-order, at the earliest possible offset where the link is idle and the queue is empty. If the frame is not assigned to the same empty-queue block as it was on the previous link, the algorithm backtracks and reschedules the previous frame to the next empty-queue block as shown in the \autoref{fig:heuristic_q1}. Condition \ref{item:flow_transmission} is enforced by using the frame offset on a particular link to set a lower bound for the offset on the next link.

Note that the the idle-link blocks are updated each time a frame is scheduled, but the empty-queue blocks remain the same for all frames in the flow.
This prevents frames in the same flow from overlapping in the time domain, but allows them to be queued simultaneously.

The heuristic succeeds if the last frame is scheduled within the deadline and fails otherwise. In \autoref{fig:heuristic_q1} the last frame spans beyond the end of the period. Assuming $\s.D \leq \s.T$ this means that it is impossible to schedule \s[2] on $q_1$ when \s[1] is arranged as in \autoref{fig:heuristic_q1}. However, assigning \s[2] to $q_2$ is straightforward because the queue is empty, i.e., all frames trivially belong to the same (only) empty-queue block as shown in \autoref{fig:heuristic_q2}.
Analogous to \gls{asap}, an \gls{alap} approach can be formulated by traversing the frames in reverse order, as well as scheduling on links in reverse order.
The reader is referred to \cite{raagaardthesis}, for more details about determining feasible frame offsets using the \gls{asap} and \gls{alap} approaches.

\subsubsection{Reducing queue usage}
To minimize queue usage, the heuristic initially assigns all flow instances to the first queue. If the heuristic at some point fails to schedule a particular frame on one of its links, it may be because the queue assignment imposes too many restrictions on the feasible frame offsets like shown in \autoref{fig:heuristic_q1}. In this case, the queue assignment is incremented for some link on the route, before restarting the algorithm from the first frame. The idle-link and empty-queue blocks are used to determine which link to increment. The \gls{asap} heuristic increments the first queue assignment which allows a frame to start earlier than with the current queue assignment.
When the algorithm terminates one of two things has happened: Either all frames have been scheduled, or some switch has no more queues available, i.e., the heuristic failed to schedule the flow.

\subsubsection{Reducing end-to-end latency}
Once a feasible solution has been found it can be post-processed to minimize end-to-end latency. Recall, that the end-to-end latency is the time from the offset of the first frame on the first link and until the finish time of the last frame on the last link. Hence, shifting the first frame to the right, or the last frame to the left reduces end-to-end latency. The intervals in which frames can safely be shifted without violating feasibility are computed from the empty-queue and idle-link intervals. A frame can be post-processed immediately when it has been scheduled on all links, or all frames can be post-processed together when the entire flow has been scheduled.

\autoref{fig:heuristic_variations} shows heuristic variations including the original \gls{asap} heuristic~(\autoref{fig:asap}). White boxes represent the intervals where frames can be shifted.
The main variation, ASAPQ, is illustrated in \autoref{fig:asapq}. It reduces the time frames spend in queues by shifting each frame instance as close as possible to the frame instance on the next link. Frame instances on the last link are not moved, which is illustrated with a thick border in \autoref{fig:asapq}. As an important side effect, the method reduces the overall time the queue is occupied, which could lead to better queue utilization and a lower total number of queues.

The remaining \gls{asap} variations are different ways of post-proces\-sing the offsets once all frames have been scheduled by either \gls{asap} or ASAPQ. In ASAP-L (\autoref{fig:asap_l}) all frame instances are shifted toward the last frame instance to reduce end-to-end latency. The schedule produced by ASAP-LF (\autoref{fig:asap_lf}) has been through an additional post-processing step, where all frames instances are shifted toward the first frame instance. ASAPQ-L and ASAPQ-LF are variations of ASAPQ that have been post-processed in the same two ways.

The same variations can be formulated for the \gls{alap} heuristic, but every shift is reversed compared to \gls{asap}. Consequently, the post-processing steps first move toward the \emph{first} frame instance, then the \emph{last}. \autoref{fig:alap}--\ref{fig:alapq_lf} depict the variations for \gls{alap}.

\begin{figure*}[t]
	\centering
	\begin{subfigure}{0.185\linewidth}
      \centering
	   \includegraphics[height=2cm]{figures/small_topologies}
	   \caption{Small; G1, G2, G3}
      \label{fig:small_topologies}
	\end{subfigure}
	\centering
	\begin{subfigure}{0.45\linewidth}
      \centering
	   \includegraphics[height=2cm]{figures/medium_topologies}
	   \caption{Medium; G4, G5}
      \label{fig:medium_topologies}
	\end{subfigure}
	\begin{subfigure}{0.3\linewidth}
      \centering
	   \includegraphics[height=2cm]{figures/topology_G7}
	   \caption{Large; G6}
      \label{fig:large_topology}
	\end{subfigure}
%\vspace{3ex}
   \caption{Topologies for experimental evaluation}
   \label{fig:topologies}
\vspace{-1ex}
\end{figure*}

\section{Experimental Evaluation}

In the first set of experiments we were interested to evaluate the ability of our \gls{grasp}-based approach to obtain near-optimal results in a reasonable time.
\begin{modifiedparagraph}
Thus, we have implemented the OMT approach from~\cite{craciunas16} and the ILP approach from~\cite{pop16}, which both minimize the number of queues ($K$). The values of $K$ are presented in the table, including the lower and upper bounds, and the normalized value. We have compared the three approaches on the test cases from~\cite{pop16}, and our GRASP has been able to obtain the same optimal solutions in under 1 second for all test cases as shown in \autoref{tab:ietcps_comparison}.
\end{modifiedparagraph}
\begin{table}[t]
   \small
\centering
\begin{modifiedparagraph}
\begin{tabular}{r|rrr|cccc}
    & \multicolumn{3}{c|}{\textit{running time (s)}} & \multicolumn{4}{c}{\textit{queue usage}} \\
ID  & ILP            & OMT           & GRASP         & $K$   & $\lbound{K}$   & $\ubound{K}$  & $K_N$  \\ \hline
T01 & 0.66           & 0.81          & 0.32          & 2     & 2              & 5             & 0      \\
T04 & 2.49           & 2.46          & 0.21          & 2     & 2              & 5             & 0      \\
T05 & 3.73           & 3.43          & 0.34          & 2     & 2              & 3             & 0      \\
T10 & 4.70           & 5.12          & 0.72          & 4     & 4              & 8             & 0      \\
T11 & 16.54          & 12.94         & 0.84          & 3     & 3              & 7             & 0      \\
T12 & 210.03         & 34.33         & 0.69          & 5     & 5              & 9             & 0      \\
T14 & 39.06          & 22.87         & 0.84          & 2     & 2              & 3             & 0      \\
T18 & 10.98          & 7.17          & 0.56          & 2     & 2              & 5             & 0     
\end{tabular}
\end{modifiedparagraph}
\caption{\modifiedtext{Comparison of ILP, OMT, and GRASP}}
\label{tab:ietcps_comparison}
\end{table}

The proposed \gls{grasp} approach is envisioned to be used as scheduling algorithm inside a configuration agent~\cite{gutierrez2015} which monitors and reschedules (parts of) the network as needed at runtime. Thus, in the second set of experiments we were interested to evaluate the ability of \gls{grasp} to find solutions in a very limited time for large test cases. 

%We evaluate the proposed strategy on several test cases.
%Generating interesting test cases is a problem in itself. If there is too much space in a schedule, every flow is trivially scheduled with low end-to-end latency and using only a single queue in each egress port. However, if the traffic of a test case is too congested, there may not exist any feasible solutions for the scheduling problem.
We consider six topologies of varying size. The topologies are industrial sized, and are derived from the work presented in \cite{oliver14}. The topologies are grouped into three categories based on their size. There are three small topologies, two medium, and one large, shown in \autoref{fig:topologies}. Blue squares illustrate end systems and green circles illustrate switches.
%Note, that there are no redundant routes in the topologies, i.e., there exists only one route between any pair of end systems.
The network precision is assumed to be $\delta = \mus{5.008}$. The transmission rate for all links is fixed at 1 Gbps, and the propagation delay of each link is assumed negligible, i.e., it is set to zero. Every egress port has eight queues.

The hyperperiod of all flows defines the width of the schedule, and has a major impact on the complexity of the problem. Thus, the hyperperiod is an important aspect to consider, when evaluating performance. We define three hyperperiods of \ms{1}, \ms{6}, and \ms{30}. For each choice of hyperperiod we define a set of short periods and a set of long periods as presented in~\autoref{tab:periods}.
Short-period flows have a data size of either one, two, or three times the \gls{mtu} of 1500 bytes. Long-period flows have data sizes 10, 20, 40, 60, or 100 times \gls{mtu}. The choice of periods and data sizes are inspired by~\cite{craciunas16}.

In order to generate flows, that yield difficult scheduling problems in terms of queue usage and end-to-end latencies, the link utilization should be relatively high. Hence, synthetic applications are generated by repeatedly adding short-period and long-period flows to the set of flows. The sending and receiving end systems are chosen at random among the end systems in the topology. This procedure is repeated until multi-queue scenarios arise.
%The \gls{asap} heuristic is used as a proxy function to indicate when the network is sufficiently saturated with flows.
%, i.e., all test cases are schedulable by the \gls{asap} heuristic, and hence, at least one feasible solution exists.
For each choice of topology and hyperperiod, we generate 30 test cases with high link utilization. In total we use 540 test cases, 90 for each of the six topologies.

\autoref{tab:test_cases_stats} shows the average number of flows and frames for every pair of topology class and hyperperiod. Overall, the test instances range from a few hundred frames to tens of thousands of frames.
\begin{table}[b]
   \small
\centering
\begin{tabular}{lll}
hyperperiod & short periods                 & long periods   \\ \hline
\ms{1}      & 100, 200, \mus{500}           & \ms{1}         \\
\ms{6}      & 100, 150, 200, \mus{500}      & 1, 2, \ms{6}   \\
\ms{30}     & 100, 150, 200, 300, \mus{500} & 5, 10, \ms{30}
\end{tabular}
\caption{Combinations of periods in test cases}
\label{tab:periods}
\vspace{0.2cm}
\centering
\begin{tabular}{llll}
            & \multicolumn{3}{c}{\emph{topology size}}            \\
   hyperperiod  & small       & medium        & large          \\ \hline
   \ms{1}  & $(17, 174)$ & $(61, 548)  $ & $(358, 2078)$  \\
   \ms{6}  & $(15, 436)$ & $(55, 1193)$  & $(254, 3682)$  \\
   \ms{30} & $(18, 737)$ & $(63, 2944)$  & $(327, 15167)$
\end{tabular}
\caption{Average number of (flows, frames) of test cases}
\label{tab:test_cases_stats}
\vspace{-0.2cm}
\end{table}

We wish to experimentally evaluate GRASP in terms of its ability to minimize queue usage as well as end-to-end latency.
\begin{modifiedparagraph}
To this end, we introduce two objective function configurations $f_K(x) = K_N(x)$ and $f_\Lambda(x) = \Lambda_N(x)$ corresponding to $c_1 = 1, c_2 = 0$ and $c_1 = 0, c_2 = 1$, respectively (see \autoref{sec:objective_function}).
\end{modifiedparagraph}

%\begin{figure}[b]
%\centering
%	\begin{subfigure}{\linewidth}
%        \centering
%        \includegraphics[width=\linewidth]{figures/plot1a}
%        \vspace{-0.6cm}
%        \caption{Queue usage for individual instances}
%        \vspace{0.4cm}
%        \label{fig:ilp_grasp_q}
%   \end{subfigure}
%	\begin{subfigure}{\linewidth}
%        \centering
%        \includegraphics[width=\linewidth]{figures/plot1b}
%        \vspace{-0.6cm}
%        \caption{End-to-end latency for individual instances}
%        %\vspace{0.4cm}
%        \label{fig:ilp_grasp_e2e}
%   \end{subfigure}
%   \caption{Comparison of GRASP and ILP} \label{fig:ilp_grasp}
%\end{figure}

\autoref{fig:grasp_heuristic_q} and \autoref{fig:grasp_heuristic_e2e} show a comparison of GRASP and the heuristic approach.
For the heuristic approach, each test case is solved using all heuristic variations (\autoref{fig:heuristic_variations}) and then the best schedule is chosen.
\begin{modifiedparagraph}
\autoref{fig:grasp_heuristic_q} shows normalized values for queue usage ($K_N$) and \autoref{fig:grasp_heuristic_e2e} shows normalized values for end-to-end latency ($\Lambda_N$). The visualizations are based on average values for all topologies and hyperperiods (540 test cases in total).
\end{modifiedparagraph}
GRASP improves queue usage as well as end-to-end latency for all topologies and hyperperiods compared to the heuristic.
\modifiedtext{The average reduction is 40\% for $K_N$ and 33\% for $\Lambda_N$.}
The heuristic has an average execution time of 28 seconds compared to 15 minutes for GRASP.

\begin{modifiedparagraph}
Furthermore, we have extended the ILP formulation presented in \cite{pop16}, which minimizes queue usage, to also feature end-to-end latency minimization.
For the ILP formulation, the Gurobi~\cite{gurobi} solver was given a time limit of 4 hours, after which it returns the best feasible solution.
The ILP approach is intractable for many of the test instances, especially for larger hyperperiods.
The results are compared with GRASP in \autoref{fig:grasp_ilp_q} and \autoref{fig:grasp_ilp_e2e} for the subset of test cases solved by ILP.
Some data points are missing, because the ILP approach was unable to find feasible solutions within the time limit. The ILP approach was able to solve 48\% of the instances when minimizing queue usage and 42\% when minimizing latency.
On average, the ILP approach produced schedules with 17\% lower queue usage in \autoref{fig:grasp_ilp_q} and 51\% lower end-to-end latency in \autoref{fig:grasp_ilp_e2e}, but had a 15--20 times longer execution time.
\end{modifiedparagraph}

\begin{figure}[t]
\centering
	\begin{subfigure}{\linewidth}
        \centering
        \includegraphics[width=\linewidth]{figures/plot2a}
        \caption{\modifiedtext{Average queue usage for GRASP and heuristic}}
        \label{fig:grasp_heuristic_q}
   \end{subfigure}
	\begin{subfigure}{\linewidth}
        \centering
        \includegraphics[width=\linewidth]{figures/plot2b}
        \caption{\modifiedtext{Average end-to-end latency for GRASP and heuristic}}
        \label{fig:grasp_heuristic_e2e}
   \end{subfigure}
	\begin{subfigure}{\linewidth}
        \centering
        \includegraphics[width=\linewidth]{figures/plot4a}
        \caption{\modifiedtext{Average queue usage for GRASP and ILP}}
        \label{fig:grasp_ilp_q}
   \end{subfigure}
	\begin{subfigure}{\linewidth}
        \centering
        \includegraphics[width=\linewidth]{figures/plot4b}
        \caption{\modifiedtext{Average end-to-end latency for GRASP and ILP}}
        \label{fig:grasp_ilp_e2e}
   \end{subfigure}
   \caption{\modifiedtext{Comparison of GRASP, heuristic and ILP}}
\label{fig:grasp_ilp_heuristic}
\end{figure}

%The ILP approach is intractable for large instances, hence, where data points are missing, the ILP approach was unable to find feasible solutions within the time limit.

\gls{grasp} is able to significantly improve execution time compared to the \gls{ilp} approach which is intractable for large instances, and is able to produce better schedules than a pure heuristic approach. Its ability to minimize the objectives could be improved by increasing the time limit. Conversely, the time limit can be decreased in order to compute feasible schedules quickly. This flexibility makes \gls{grasp} well-suited to be used for runtime reconfiguration, where the schedule must adapt to changes in network traffic. 
\section{Conclusions}
In this paper, we have addressed the configuration of \gls{tsn} networks in safety-critical real-time systems.
We have formulated the problem of scheduling periodic \glsfirst{tt} flows by means of \glsfirstplural{gcl} as defined in the \IEEE{802.1Qbv} standard.
The configuration of \glsplural{gcl} corresponds to scheduling the temporal offset of each frame, and assigning flows to queues in egress ports of switches.

We have proposed a \gls{grasp}-based optimization strategy for synthesizing schedules with minimal queue usage and end-to-end latency.
%The strategy consists of a constructive phase, which relies on underlying heuristics for arranging individual frames, and an exploration phase, which searches for improving solutions by rescheduling individual frames and reassigning flows to egress port queues.
%This results in quickly discovering feasible solutions, which are gradually improved over time.
The proposed metaheuristic strategy is capable of scheduling very large instances, which are intractable with previously proposed methods such as ILP and SMT. Furthermore, the strategy is well-suited for yielding the best possible configuration within a strict time limit, e.g., related to runtime reconfiguration.

%\vspace{-0.1cm}

\bibliographystyle{ACM-Reference-Format}
\bibliography{sigproc}

\end{document}
